\input{/home/reece/Dropbox/UMKC/Notes/LaTeXPreamble/preamble}
\usepackage{systeme}
\usepackage{color, colortbl}
\usepackage{graphicx}
\usepackage{hyperref}
\hypersetup{
    colorlinks,
    citecolor=black,
    filecolor=black,
    linkcolor=black,
    urlcolor=black
}
\graphicspath{ {/} }
\definecolor{LightCyan}{rgb}{0.88,1,1}
\title{Math 300 - Final Exam Study Guide}
\author{Reece McMillin}
\begin{document}
\maketitle
\tableofcontents
\large
\hyperlink{cramer}{1. Cramer's Rule}

\hyperlink{detarea}{2. Determinants as Area}

\hyperlink{eigen}{3. Eigenvalues \& Eigenvectors}

\hyperlink{diag}{4. The Diagonalization Theorem}

\hyperlink{imt}{5. Invertible Matrix Theorem}

\hyperlink{compeig}{6. Complex Eigenvalue Decomposition}
\pagebreak
\normalsize
\begin{defbox}{Cramer's Rule}
    \hypertarget{cramer}
    \[
        \mathbf{A}_i(b) = \begin{bmatrix}a_1, \dots, b, \dots, a_2\end{bmatrix}
    \]
    Let $\mathbf{A}$ be an invertible $n \times n$ matrix. For any $b \in \mathbb{R}^n$, the unique solution $\vec{x}$ of $\mathbf{A}\vec{x} = b$ has entries given by
    \[
    \vec{x}_i = \frac{\det{\mathbf{A}_i(b)}}{\det{\mathbf{A}}}
    \]
    An inverse formula extending from Cramer's Rule is
    \[
        \mathbf{A}^{-1} = \frac{1}{\det{A}} \cdot \text{adj}(\mathbf{A})
    \]
\end{defbox}
\begin{defbox}{Determinants as Area}
    \hypertarget{detarea}
    If $\mathbf{A}$ is a $2\times2$ matrix, the area of the parallelogram determined by the columns of $\mathbf{A}$ is $\left|\det{A}\right|$. If $\mathbf{A}$ is a $3\times3$ matrix, the volume of the parallelopiped determined by the columns of $\mathbf{A}$ is $\left|\det{A}\right|$.
\end{defbox}
\begin{defbox}{Eigenvalues \& Eigenvectors}
    \hypertarget{eigen}
    An \textbf{eigenvector} of $\mathbf{A}$ (corresponding to $\lambda$) is a nonzero vector x such that 
    \[
        \mathbf{A}x = \lambda x
    \]
    A scalar $\lambda$ is an \textbf{eigenvalue} of $\mathbf{A}$ if there exists a nonzero vector $x$ such that
    \[
        \mathbf{A}x = \lambda x
    \]
    More formally, let $\mathbf{V}$ be a vector space. An \textbf{eigenvector} of a linear transformation $\mathbf{T} : \mathbf{V} \to \mathbf{V}$ is a nonzero vector $x \in \mathbf{V}$ such that $\mathbf{T}(x) = \lambda x$ for some scalar $\lambda$. This scalar $\lambda$ is called an \textbf{eigenvalue} of $\mathbf{T}$ if there is a nontrivial solution $x$ of $\mathbf{T}(x) = \lambda x$; ssuch an $x$ is called an \textbf{eigenvector} corresponding to $\lambda$.
\end{defbox}
\begin{defbox}{The Diagonalization Theorem}
    \hypertarget{diag}
    An $n\times n$ matrix $\mathbf{A}$ is diagonalizable if and only if $\mathbf{A}$ has $n$ linearly independent eigenvectors.\\\\
    In fact, $\mathbf{A} = \mathbf{PDP}^{-1}$, with diagonal matrix $\mathbf{D}$, if and only if the columns of $\mathbf{P}$ are $n$ linearly independent eigenvectors of $A$. In this case, the diagonal entries of $\mathbf{D}$ are eigenvalues of $\mathbf{A}$ that correspond respectively to the eigenvectors in $\mathbf{P}$.\\\\
    \textbf{Theorem: }An $n\times n$ matrix with $n$ distinct eigenvalues is diagonalizable.
\end{defbox}
\pagebreak
\begin{defbox}{\hypertarget{imt} Invertible Matrix Theorem}
    \begin{enumerate}
        \item $\mathbf{A}$ is row-equivalent to the $n\times n$ identity matrix $\mathbf{I}_n$.
        \item $\mathbf{A}$ has $n$ pivot positions.
        \item The equation $\mathbf{A}x = 0$ has only the trivial solution $x = 0$.
        \item The columns of $\mathbf{A}$ form a linearly independent set.
        \item The linear transformation $x \mapsto \mathbf{A}x$ is one-to-one.
        \item For each column vector $b \in \mathbb{R}^n$, the equation $\mathbf{A}x = b$ has a unique solution.
        \item The columns of $\mathbf{A}$ span $\mathbb{R}^n$.
        \item The linear transformation $x \mapsto \mathbf{A}x$ is a surjection (onto).
        \item There is an $n\times n$ matrix $\mathbf{C}$ such that $\mathbf{C}\mathbf{A} = \mathbf{I}_n$.
        \item There is an $n\times n$ matrix $\mathbf{D}$ such that $\mathbf{A}\mathbf{D} = \mathbf{I}_n$.
        \item The transpose matrix $\mathbf{A}^\text{T}$ is invertible. 
        \item The columns of $\mathbf{A}$ form a basis for $\mathbb{R}^n$.
        \item The column space of $\mathbf{A}$ is equal to $\mathbb{R}^n$.
        \item The dimension of the column space of $\mathbf{A}$ is $n$.
        \item The rank of $\mathbf{A}$ is $n$.
        \item The null space of $\mathbf{A}$ is ${0}$.
        \item The dimension of the null space of $\mathbf{A}$ is $0$.
        \item $0$ fails to be an eigenvalue of $\mathbf{A}$.
        \item The determinant of $\mathbf{A}$ is not $0$.
        \item The orthogonal complement of the column space of $\mathbf{A} = \mathbf{A}^\perp$ is ${0}$.
        \item The orthogonal complement of the null space of $\mathbf{A}$ is $\mathbb{R}^n$.
        \item The row space of $\mathbf{A}$ is $\mathbb{R}^n$.
        \item The matrix $\mathbf{A}$ has $n$ non-zero singular values (\textit{not studied in MATH-300}).
    \end{enumerate}
\end{defbox}
\pagebreak
\begin{defbox}{\hypertarget{compeig}Complex Eigenvalue Decomposition}
    Let $\textbf{A}$ be a real $2 \times 2$ matrix with a complex eigenvalue $\lambda = a - bi$ $(b \neq 0)$ and an associated eigenvector $v \in \mathbb{C}^2$. Then
    \[
        \mathbf{A} = \mathbf{PCP}^{-1} \text{, where } \mathbf{P} = \begin{bmatrix}\text{Re }v & \text{Im } v\end{bmatrix} \text{ and } \mathbf{C} = \begin{bmatrix}a & -b \\ b & a\end{bmatrix}
    \]
\end{defbox}
\begin{defbox}{Inner Product}
    \[
        \mathbf{u \cdot v = u^Tv}
    \]
    Let $\mathbf{u}, \mathbf{v}, \mathbf{w}$ be vectors in $\mathbb{R}^n$ and let $c$ be a scalar. Then
    \begin{enumerate}
        \item $\mathbf{u \cdot v} = \mathbf{v \cdot u}$
        \item $(\mathbf{u} + \mathbf{v}) \cdot \mathbf{w} = \mathbf{u} \cdot \mathbf{w} + \mathbf{v} \cdot \mathbf{w}$
        \item $(c\mathbf{u}) \cdot \mathbf{v} = (c\mathbf{u \cdot v}) = \mathbf{u} \cdot (c\mathbf{v})$ 
        \item $\mathbf{u \cdot u} \geq 0$, and $\mathbf{u \cdot u} = 0$ if and only if $\mathbf{u} = 0$.
    \end{enumerate}
\end{defbox}
\begin{defbox}{Distance}
    For $\mathbf{u}$ and $\mathbf{u} \in \mathbb{R}^n$, the \textbf{distance between $\mathbf{u}$ and $\mathbf{v}$}, written as $\text{dist}(\mathbf{u}, \mathbf{v})$, is the length of the vector $\mathbf{u} - \mathbf{v}$. That is,
    \[
        \text{dist}(\mathbf{u}, \mathbf{v}) = ||\mathbf{u} - \mathbf{v}||
    \]
\end{defbox}
\begin{defbox}{Orthogonality}
    Two vectors $\mathbf{u}, \mathbf{v} \in \mathbb{R}^n$ are \textbf{orthogonal} to each other if $\mathbf{u \cdot v} = 0$.\\\\
    Two vectors $\mathbf{u}, \mathbf{v} \in \mathbb{R}^n$ are \textbf{orthogonal} to each other if $||\mathbf{u} + \mathbf{v}||^2 = ||\mathbf{u}||^2 + ||\mathbf{v}||^2$.
    \subsubsection*{Orthogonal Complements}
    \begin{enumerate}
        \item A vector $x$ is in $\mathbf{W}^\perp$ if and only if $x$ is orthogonal to every vector in a set that spans $\mathbf{W}$.
        \item $\mathbf{W}$ is a subspace of $\mathbb{R}^n$.
    \end{enumerate}
    \textbf{Theorem:} Let $\mathbf{A}$ be an $m \times n$ matrix. The orthogonal complement of the row space of $\mathbf{A}$ is the null space of $\mathbf{A}$, and the orthogonal complement of the column space of $\mathbf{A}$ is the null space of $\mathbf{A}^T$:
    \[
        (\text{Row }\mathbf{A})^\perp = \text{Null }\mathbf{A} \text{\quad and\quad} (\text{Col }\mathbf{A})^\perp = \text{Null }\mathbf{A}^\perp
    \]
    To find the angle between two nonzero vectors in either $\mathbb{R}^2$ or $\mathbb{R}^3$, the inner product can be used:
    \[
        \mathbf{u \cdot v} = ||\mathbf{u}|| ||\mathbf{v}|| \cos{\theta}
    \]
\end{defbox}
\pagebreak
\begin{defbox}{Orthogonal Sets}
    \textbf{Theorem: } If $S = {u_1, \dots, u_p}$ is an orthogonal set of nonzero vectors in $\mathbb{R}^n$, then $S$ is linearly independent and hence is a basis for the subspace spanned by $S$.\\\\
    An \textbf{orthogonal basis} for a subspace $\mathbf{W}$ of $\mathbb{R}^n$ is a basis for $\mathbf{W}$ that is also an orthogonal set.\\\\
    \textbf{Theorem: } Let $S = {u_1, \dots, u_p}$ be an orthogonal basis for a subspace $\mathbf{W}$ of $\mathbb{R}^n$. For each $y \in \mathbf{W}$, the weights in the linear combination
    \[
        y = c_1u_1 + \dots + c_pu_p
    \]
    are given by
    \[
        c_j = \frac{y \cdot u_j}{u_j \cdot u_j}\quad(j = 1,\dots,p)
    \]
\end{defbox}
\begin{defbox}{Orthogonal Projection}
    Given some nonzero vector $u$ in $\mathbb{R}^n$, consider the problem of decomposing a vector $y \in \mathbb{R}^n$ into the sum of two vectors, one a multiple of $u$ and the other orthogonal to $u$. We want to write
    \[
        y = \hat{y} + z,\quad z \perp u
    \]
    We can consider $\hat{y}$ to be the shadow of $y$ onto $W$ $(\hat{y} = \text{proj}_W{y})$ and z to be the remaining vertical component of $y$. This projection is determined by the \textit{subspace} $L$ spanned by $u$ (the line through $u$ and 0). $\hat{y}$ is denoted by $\text{proj}_Ly$ and is called the \textbf{orthogonal projection of $y$ onto $L$}. That is,
    \[
        \hat{y} = \text{proj}_L{y} = \frac{y\cdot u}{u\cdot u}u
    \]
\end{defbox}
\begin{defbox}{The Best Approximation Theorem}
    Let $\mathbf{W}$ be a subspace of $\mathbb{R}^n$, let $y$ be any vector in $\mathbb{R}^n$, and let $\hat{y}$ be the orthogonal projection of $y$ onto $\mathbf{W}$. Then $\hat{y}$ is the closest point in $\mathbf{W}$ to $y$, in the sense that
    \[
        ||y - \hat{y}|| < ||y - v||
    \]
    for all $v \in \mathbf{W}$ distinct from $\hat{y}$.
\end{defbox}
\begin{defbox}{The Gram-Schmidt Process}
    Given a basis ${x_1, \dots, x_p}$ for a nonzero subspace $\mathbf{W}$ of $\mathbb{R}^n$, define
    \begin{align*}
        v_1 &= x_1\\
        v_2 &= x_2 - \frac{x_2 \cdot v_1}{v_1\cdot v_1}v_1\\
        v_3 &= x_3 - \frac{x_3 \cdot v_1}{v_1\cdot v_1}v_1 - \frac{x_3 \cdot v_2}{v_2\cdot v_2}v_2\\
        &\vdots\\
        v_p &= x_p - \frac{x_p \cdot v_1}{v_1\cdot v_1}v_1 - \frac{x_p \cdot v_2}{v_2\cdot v_2}v_2 - \dots - \frac{x_p \cdot v_{p-1}}{v_{p-1}\cdot v_{p-1}}v_{p-1}
    \end{align*}
    Then ${v_1, \dots, v_p}$ is an orthogonal basis for $\mathbf{W}$. In addition,
    \[
        \text{Span}\{v_1, \dots, v_k\} = \text{Span}\{x_1, \dots, x_k\}\quad\quad \text{for } 1 \leq k \leq p
    \]
\end{defbox}
\end{document}